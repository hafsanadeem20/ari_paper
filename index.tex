% !TeX root = RJwrapper.tex
\title{The Automated R Instructor}
\author{by Sean Kross, John Muschelli, Jeffrey T. Leek}

\maketitle

\abstract{%
An abstract of less than 150 words.
}

% Any extra LaTeX you need in the preamble

\hypertarget{introduction}{%
\subsection{Introduction}\label{introduction}}

Videos are a large way people learn. Creating videos of an speaker with
slides take time, energy, and usually a bit if video editing skills. A
large issue with such videos is that updating the materials either
requires remaking the entire video or extensive editing and splicing of
new segments. We present \CRANpkg{ari}, the automated R instructor to
mitigate these issues by creating reproducible presentations and videos
that can be automatically generated. By using Ari we hope to be able to
rapidly create and update video content.

Multiple open source tools for video editing and splicing exist. The
\texttt{ffmpeg} software is highly powerful, has been thoroughly tested,
and has been developed for over XX years.

\hypertarget{technical-stuff}{%
\subsection{Technical stuff}\label{technical-stuff}}

The \pkg{ari} package relies on \href{https://ffmpeg.org/}{FFmpeg}
(\textgreater{}= 3.2.4) to interleave the images and the audio files.

\hypertarget{amazon-authetication}{%
\subsection{Amazon authetication}\label{amazon-authetication}}

The \pkg{ari} package relies on the \CRANpkg{tuneR} package for reading
and manipulating audio files such as MP3 and WAV files. The voice
synthesis is done by Amazon Polly, which is a text to speech voice
generation engine with over XX languages, implemented in the
\CRANpkg{aws.polly} package. In addition to multiple languages, Amazon
Polly provides voices of different gender within the same language, as
well as different dialects such as American English.

In order to use Amazon Polly, you must set up R to use Amazon Web
Services, which we have provided as a tutorial
\href{http://seankross.com/2017/05/02/Access-Amazon-Web-Services-in-R.html}{here}.
In order to test the authentication, you can run
\texttt{aws.polly::list\_voices()}.

\hypertarget{making-videos-with-ari}{%
\subsection{\texorpdfstring{Making videos with
\texttt{ari}}{Making videos with ari}}\label{making-videos-with-ari}}

Videos can be produced form \texttt{ari} using an HTML slide
presentation based in RMarkdown, where the script is in the HTML
comments, or a simple vector of images and paragraphs.

The main function is \texttt{ari\_spin}, which takes in a set of images
and a series of paragraphs of text. These paragraphs are the ``script''
that is spoken over the images to create the output video. The number of
paragraphs need to be equal to the number of images. The output video
format is MP4 by default, but can be any format that the \texttt{ffmpeg}
installation codecs support. Supported codecs can be founded using the
functions \texttt{ffmpeg\_audio\_codecs} and
\texttt{ffmpeg\_video\_codecs}.

The main workhorse of \pkg{ari} is the \texttt{ari\_stitch} function.
This function takes in a vector of images and a series of Wav audio
objects or audio filenames.\\
We have tested the videos on YouTube and the Coursera platform.
Additional video specifications can be applied to \texttt{ffmpeg\_opts}
argument of \texttt{ari\_stitch}.

\hypertarget{accessibility}{%
\subsection{Accessibility}\label{accessibility}}

With respect to accessibility, as \pkg{ari} has the direct script that
was spoken, this provides for direct subtitles for those hard of hearing
rather than relying on other services, such as YouTube, to provide a
speech to text translation. Though some changes to the script are
required for AMazon Polly to correctly pronounce the information, these
can be changed using regular expressions in the script, and then passed
to \texttt{ari\_subtitles}.

\hypertarget{future-directions}{%
\subsection{Future directions}\label{future-directions}}

We believe the heavy reliance on an \texttt{ffmpeg} installation can be
mitigated in the future with advances in the \pkg{av} package. Though
the \pkg{av} package has powerful functionality and is currently porting
more from \texttt{libav} and therefore \texttt{ffmpeg}, it currently
does not have the capabilities requried for \pkg{ari}. Although third
party installation from \url{https://ffmpeg.org/} can be burdensome to a
user, package managers such as \texttt{brew} for OSX and \texttt{choco}
for Windows provide installations.

Although we rely on Amazon Polly for voice synthesis, other packages
provide voice synthesis, such as \CRANpkg{mscstts} for Microsoft and
\CRANpkg{googleLanguageR} for Google. We aim to harmonize these
synthesis options, so that users can choose to create videos with the
services that they support or have access to.

Scripts can be automatically translated into other languages with
services like the \href{https://cloud.google.com/translate/docs/}{Google
Translation API}, which \pkg{googleLanguageR} provides an interface.
Amazon Polly can speak languages other than English. This means you can
write a lecture once and generate slides and videos in multiple
languages.

\hypertarget{examples-from-readme---edit}{%
\subsection{Examples (FROM README -
edit)}\label{examples-from-readme---edit}}

These examples make use of the \texttt{ari\_example()} function. In
order to view the files mentioned here you should use
\texttt{file.show(ari\_example("{[}file\ name{]}"))}. You can watch an
example of a video produced by Ari
\href{https://youtu.be/dcIUu4GCOKU}{here}.

\begin{verbatim}
library(ari)

# First set up your AWS keys
Sys.setenv("AWS_ACCESS_KEY_ID" = "EA6TDV7ASDE9TL2WI6RJ",
           "AWS_SECRET_ACCESS_KEY" = "OSnwITbMzcAwvHfYDEmk10khb3g82j04Wj8Va4AA",
           "AWS_DEFAULT_REGION" = "us-east-2")

# Create a video from a Markdown file and slides
ari_narrate(
  ari_example("ari_intro_script.md"),
  ari_example("ari_intro.html"),
  voice = "Joey")

# Create a video from an R Markdown file with comments and slides
ari_narrate(
  ari_example("ari_comments.Rmd"),
  ari_example("ari_intro.html"),
  voice = "Kendra")

# Create a video from images and strings
ari_spin(
  ari_example(c("mab1.png", "mab2.png")),
  c("This is a graph.", "This is another graph"),
  voice = "Joanna")

# Create a video from images and Waves
library(tuneR)
ari_stitch(
  ari_example(c("mab1.png", "mab2.png")),
  list(noise(), noise()))
\end{verbatim}

\hypertarget{rmarkdownhtml-slide-problems}{%
\subsubsection{RMarkdown/HTML slide
Problems}\label{rmarkdownhtml-slide-problems}}

Some html slides take a bit to render on webshot, and can be dark gray
instead of white. If you change the \texttt{delay} argument in
\texttt{ari\_narrate}, passed to \texttt{webshot}, this can resolve some
issues, but may take a bit longer to run. Also, using
\texttt{capture\_method\ =\ "vectorized"} is faster, but may have some
issues, so run with \texttt{capture\_method\ =\ "iterative"} if this is
the case as so:

\begin{verbatim}
ari_narrate(
  ari_example("ari_comments.Rmd"),
  ari_example("ari_intro.html"),
  voice = "Kendra",
  delay = 0.5,
  capture_method = "iterative")
\end{verbatim}

\bibliography{RJreferences}


\address{%
Sean Kross\\
UCSD\\
line 1\\ line 2\\
}
\href{mailto:author1@work}{\nolinkurl{author1@work}}

\address{%
John Muschelli\\
Affiliation\\
line 1\\ line 2\\
}
\href{mailto:author2@work}{\nolinkurl{author2@work}}

\address{%
Jeffrey T. Leek\\
Affiliation\\
line 1\\ line 2\\
}
\href{mailto:author2@work}{\nolinkurl{author2@work}}

